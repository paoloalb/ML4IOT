\documentclass{article}
\usepackage[utf8]{inputenc}
\usepackage[english]{babel}
\usepackage[document]{ragged2e}
\usepackage{todonotes}
\usepackage[a4paper, total={6.2in, 10in}]{geometry}
\usepackage{subcaption}
\usepackage{graphicx}
\usepackage{grffile}
\usepackage{tabularx,booktabs}
\newcolumntype{Y}{>{\centering\arraybackslash}X}
\usepackage{float}
\usepackage{tabularx}
    \newcolumntype{L}{>{\raggedright\arraybackslash}X}
    
    

\date{}
\begin{document}
\author{Paolo Aberto\\
StudentID : s278098\\

\and
Lorenzo De Nisi\\
Student ID: s276545\\

\and
Carmine De Stefano\\
Student ID: s278176\\
}

\justifying

\title{
    \vspace{.5cm}
    
    Machine Learning for IoT - 
    Politecnico di Torino\\
    \vspace{1cm}
    \Large \textbf{Homework 2 report}
    \vspace{.5cm}
}

\maketitle
\thispagestyle{empty} 
\vspace{-0.5cm}


\section{Multi-Step Temperature and Humidity Forecasting}
After implementing a WindowGenerator class returning 6 output steps, we tried to train the models. We settled on a weights-only quantization, and after tweaking the pruning parameters we found two versions of the CNN model that were able to respect the given constraints.
The parameters used for the 2 models were the final sparsity (\%) at which pruning ends, and the width multiplier $\alpha$.
\\\\
We adopted a callback function to stop the training when the validation set performances stopped to improve. We adopted the standars Keras checkpoint approach, and we monitored the mean of the two MAE metrics for the checkpoint. 
\\\\
In the following table we can see the results for the two models:

\vspace{0.2cm}

\begin{center}
\begin{tabular}{ |c|c|c|c|c|c|c| } 
\hline
\textbf{Quantization} & \textbf{Final sparsity} & \textbf{$\alpha$} &\textbf{Size} &\textbf{Compr. size} &\textbf{TempMAE}&\textbf{HumMAE} \\
\hline
Weights only & 0.745 & 0.12 & 5246 B & 1992 B & 0.4812 °C & 1.786 \% \\ 
Weights only & 0.75  & 0.07 & 4480 B & 1689 B & 0.4795 °C & 1.879 \% \\ 
\hline

\end{tabular}
\captionof{table}{Metrics for the final models obtained (version A and B)} 

\end{center}

\vspace{0.1cm}

\section{Keyword Spotting}

For keyword spotting on the mini speech command dataset, we used the DS-CNN model with weights-only quantization, obtaining the models in the table below:
\\
\begin{center}
\begin{tabular}{|c|c|c|c|c|c|c|c|}
\hline
\textbf{Quantiz.} &\textbf{F.S.} & \textbf{$\alpha$} &\textbf{Size} &\textbf{Compr.}&  \textbf{Inf. Latency}& \textbf{Tot. Latency} &\textbf{Test acc.}\\
\hline
Weights only & 0.9 & 0.7 & 77248 B & 21764 B & 9.03ms&   66.71ms & 0.91375 \\ 
Weights only & - &  0.3  & 26960 B & 23233 B  &  1.16ms& 58.72ms & 0.91 \\ 
Weights only & - &  0.4  & 40320 B & 34831 B  & 4.52ms & 21.07ms & 0.9175 \\ 

\hline
\end{tabular}
\captionof{table}{Metrics for the final models obtained (version A, B and C)} 
\end{center}

\vspace{0.5cm}

As before, we adopted a callback function to monitor the accuracy metric on the validation set.
In order to obtain the performances requested in version B, we used a quite low parameter $\alpha$, to reduce inference computation. For version C instead, we switched to STFT in order to dramatically reduce processing time.
\\\\
For both models, compression has been perormed thanks to the zlib library. Final memory performances were calculated by considering the compressed model.
For both tasks, we used the original architectures provided in the laboratory.

\end{document}
